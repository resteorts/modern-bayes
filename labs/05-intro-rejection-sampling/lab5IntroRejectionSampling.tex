\documentclass{article}
\usepackage{graphicx}
% Custom definitions
% To use this customization file, insert the line "\input{custom}" in the header of the tex file.

% Formatting

\pagenumbering{arabic} 
\setbeamertemplate{navigation symbols}{}
\setbeamertemplate{footline}[page number]

\usepackage{bbm}
% Packages
\usepackage{amssymb,amsfonts,amsmath,latexsym,amsthm}
%\usepackage[usenames,dvipsnames]{color}
%\usepackage[]{graphicx}
%\usepackage[space]{grffile}
\usepackage{mathrsfs} 
 \usepackage{amssymb,latexsym}
\usepackage{amssymb,amsfonts,amsmath,latexsym,amsthm, bm}
%\usepackage[usenames,dvipsnames]{color}
%\usepackage[]{graphicx}
%\usepackage[space]{grffile}
\usepackage{mathrsfs}   % fancy math font
% \usepackage[font=small,skip=0pt]{caption}
%\usepackage[skip=0pt]{caption}
%\usepackage{subcaption}
%\usepackage{verbatim}
%\usepackage{url}
%\usepackage{bm}
\usepackage{dsfont}
\usepackage{multirow}
\newcommand{\lam}{\lambda}
%\newcommand{\bmu}{\bm{\mu}}




% Document-specific shortcuts
\newcommand{\btheta}{{\bm\theta}}
\newcommand{\bbtheta}{{\pmb{\bm\theta}}}

\newcommand{\commentary}[1]{\ifx\showcommentary\undefined\else \emph{#1}\fi}

\newcommand{\term}[1]{\textit{\textbf{#1}}}

% Math shortcuts

% Probability distributions
\DeclareMathOperator*{\Exp}{Exp}
\DeclareMathOperator*{\TExp}{TExp}
\DeclareMathOperator*{\Bernoulli}{Bernoulli}
\DeclareMathOperator*{\Beta}{Beta}
\DeclareMathOperator*{\Ga}{Gamma}
\DeclareMathOperator*{\TGamma}{TGamma}
\DeclareMathOperator*{\Poisson}{Poisson}
\DeclareMathOperator*{\Binomial}{Binomial}
\DeclareMathOperator*{\NormalGamma}{NormalGamma}
\DeclareMathOperator*{\InvGamma}{InvGamma}
\DeclareMathOperator*{\Cauchy}{Cauchy}
\DeclareMathOperator*{\Uniform}{Uniform}
\DeclareMathOperator*{\Gumbel}{Gumbel}
\DeclareMathOperator*{\Pareto}{Pareto}
\DeclareMathOperator*{\Mono}{Mono}
\DeclareMathOperator*{\Geometric}{Geometric}
\DeclareMathOperator*{\Wishart}{Wishart}

% Math operators
\DeclareMathOperator*{\argmin}{arg\,min}
\DeclareMathOperator*{\argmax}{arg\,max}
\DeclareMathOperator*{\Cov}{Cov}
\DeclareMathOperator*{\diag}{diag}
\DeclareMathOperator*{\median}{median}
\DeclareMathOperator*{\Vol}{Vol}

% Math characters
\newcommand{\R}{\mathbb{R}}
\newcommand{\Z}{\mathbb{Z}}
\newcommand{\E}{\mathbb{E}}
\renewcommand{\Pr}{\mathbb{P}}
\newcommand{\I}{\mathds{1}}
\newcommand{\V}{\mathbb{V}}

\newcommand{\A}{\mathcal{A}}
%\newcommand{\C}{\mathcal{C}}
\newcommand{\D}{\mathcal{D}}
\newcommand{\Hcal}{\mathcal{H}}
\newcommand{\M}{\mathcal{M}}
\newcommand{\N}{\mathcal{N}}
\newcommand{\X}{\mathcal{X}}
\newcommand{\Zcal}{\mathcal{Z}}
\renewcommand{\P}{\mathcal{P}}

\newcommand{\T}{\mathtt{T}}
\renewcommand{\emptyset}{\varnothing}

\newcommand{\bmu}{\bm{\mu}}
\newcommand{\bX}   {\bm{X}}
\newcommand{\bx}   {\bm{x}}
\newcommand{\sig}   {\Sigma}
%\newcommand{\bx}{\ensuremath{\mathbf{X}}}
%\newcommand{\X}{\ensuremath{\mathbf{x}}}
%\newcommand{\w}{\ensuremath{\mathbf{w}}}
%\newcommand{\h}{\ensuremath{\mathbf{h}}}
%\newcommand{\V}{\ensuremath{\mathbf{v}}}
%\newcommand{\cov}{\text{Cov}}
\newcommand{\var}{\text{Var}}


% Miscellaneous commands
\newcommand{\iid}{\stackrel{\mathrm{iid}}{\sim}}
\newcommand{\matrixsmall}[1]{\bigl(\begin{smallmatrix}#1\end{smallmatrix} \bigr)}

\newcommand{\items}[1]{\begin{itemize} #1 \end{itemize}}

\newcommand{\todo}[1]{\emph{\textcolor{red}{(#1)}}}

\newcommand{\branch}[4]{
\left\{
	\begin{array}{ll}
		#1  & \mbox{if } #2 \\
		#3 & \mbox{if } #4
	\end{array}
\right.
}

% approximately proportional to
\def\app#1#2{%
  \mathrel{%
    \setbox0=\hbox{$#1\sim$}%
    \setbox2=\hbox{%
      \rlap{\hbox{$#1\propto$}}%
      \lower1.3\ht0\box0%
    }%
    \raise0.25\ht2\box2%
  }%
}
\def\approxprop{\mathpalette\app\relax}


\definecolor{dbrown}{rgb}{0.59, 0.29, 0.0}
\definecolor{bbrown}{rgb}{0.54, 0.2, 0.14}

% \newcommand{\approptoinn}[2]{\mathrel{\vcenter{
  % \offinterlineskip\halign{\hfil$##$\cr
    % #1\propto\cr\noalign{\kern2pt}#1\sim\cr\noalign{\kern-2pt}}}}}

% \newcommand{\approxpropto}{\mathpalette\approptoinn\relax}

\newcommand{\tth}   {\mbox{$\theta$}}
\newcommand{\thh}   {\mbox{$\theta$}}
\newcommand{\su}   {\mbox{$\sigma^2$}}
\newcommand{\so}   {\mbox{$\sigma_0^2$}}
\newcommand{\ko}   {\mbox{$\kappa_0$}}
\newcommand{\no}   {\mbox{$\nu_0$}}
\newcommand{\mo}   {\mbox{$\mu_0$}}
\newcommand{\ti}   {\mbox{$\tilde{x}$}}
\newcommand{\la}   {\mbox{$\lambda$}}
\newcommand{\bZ}   {\mbox{$\bm{Z}$}}
\newcommand{\bY}   {\mbox{$\bm{Y}$}}
\newcommand{\bA}   {\mbox{$\bm{A}$}}
\newcommand{\ba}   {\mbox{$\bm{a}$}}
\newcommand{\bb}   {\mbox{$\bm{b}$}}
\newcommand{\bt}   {\mbox{$\bm{t}$}}
\newcommand{\bz}   {\mbox{$\bm{z}$}}
\newcommand{\bw}   {\mbox{$\bm{w}$}}
\newcommand{\bbeta}   {\mbox{$\bm{\beta}$}}

\newcommand{\be}   {\mbox{$\bm{e}$}}
\newcommand{\bu}   {\mbox{$\bm{u}$}}
\newcommand{\bv}   {\mbox{$\bm{v}$}}
\newcommand{\sigx}   {\mbox{$\Sigma_{XX}$}}
\newcommand{\sigxy}   {\mbox{$\Sigma_{XY}$}}
\newcommand{\tr}   {\mbox{$\text{tr}$}}
\newcommand{\ddet}   {\mbox{$\text{det}$}}
\newcommand\independent{\protect\mathpalette{\protect\independenT}{\perp}}
\newcommand{\bya} {\bm{Y_1}}
\newcommand{\byk} {\bm{Y_K}}
\newcommand{\bxa} {\bm{X_1}}
\newcommand{\bxp} {\bm{X_p}}

\newcommand{\bxi} {\bm{x_i}}
%\newcommand{\bX} {\bm{X}}
\newcommand{\bmk} {\bm{\mu_k}}
\newcommand{\bmka} {\bm{\mu_{k_1}}}
\newcommand{\bmkb} {\bm{\mu_{k_2}}}
%\newcommand{\ba} {\bm{a}}
\newcommand{\baa} {\bm{a_1}}
\newcommand{\bak} {\bm{a_k}}
\newcommand{\bab} {\bm{a_2}}
\newcommand{\bxpp} {\bm{x^{''}}}
\newcommand{\bxap} {\bm{x_1^{'}}}
\newcommand{\bxbp} {\bm{x_2^{'}}}
\newcommand{\bxapp} {\bm{x_1^{''}}}
\newcommand{\bxbpp} {\bm{x_2^{''}}}
\newcommand{\bxkpp} {\bm{x_k^{''}}}
\newcommand{\bmkh} {\bm{\hat{\mu_k}}}
\def\hbeta{\hat{\beta}}









\begin{document}

\title{Lab 5: Introduction to Rejection Sampling in \textsf{R} - STA 360/602}
\author{Rebecca C. Steorts}
\date{}
\maketitle

%\section{Agenda}
%\begin{enumerate}
%%\item{Common errors from Lab 6}
%\item{Importance of writing well and documenting code well}
%\item{Using \textsf{dbeta}}
%\item{Using \textsf{rbeta}}
%\item{Generating a sequence using \textsf{seq}}
%\item{Plotting multiple items in the same window using \textsf{plot}}
%\end{enumerate}

\section{Agenda}

We can often end up with posterior distributions that we only know up to a normalizing constant. For example, in practice, we may derive $$p(\theta \mid x) \propto p(x\mid \theta) p(\theta)$$ and find that the normalizing constant $p(x)$ is very difficult to evaluate. Such examples occur when we start building non-conjugate models in Bayesian statistics. 

Given such a posterior, how can we appropriate it's density? One way is using rejection sampling. As an example, let's suppose our resulting posterior distribution is 
$$f(x) \propto sin^2(\pi x), x \in [0,1].$$

In order to understand how to approximate the density (normalized) of $f$, we will investigate the following tasks:

\begin{enumerate}
\item Plot the densities of $f(x)$ and the Unif(0,1) on the same plot. 
\item According to the rejection sampling approach sample from $f(x)$ using the Unif(0,1) pdf as an enveloping function.
\item Plot a histogram of the points that fall in the acceptance region. Do this for a simulation size of $10^2$ and $10^5$ and report your acceptance ratio. Compare the ratios and histograms.
\item Repeat Tasks 1 - 3 for  Beta(2,2) as an enveloping function. 
\item Compare your results with the results in Task 3.
%\item Using importance sampling calculate the $\mathbb{E}(x)$, where $x \sim f(x)$ from previous task and the proposal distribution is transformed Beta(2,2). (This will be covered in more detail in lab.)
\end{enumerate}




%\section{Directions}
%
%In general for Labs, at the top of any file you are asked to submit, please list the following:
%
%\begin{enumerate}
%\item{First Name Last Name}
%\item{Lab Date}
%\item{Team Member(s)}
%\end{enumerate}
%
%\noindent
%With respect to any item for which you are asked to generate any output, please provide the actual \textsf{R} output as a part of your solution and any explanation needed as well. For any functions/ computations that you will write, please list the following as comments before the step in \textsf{R}:
%
%\begin{enumerate}
%\item{Task number and descriptions.}
%\item{Input(s) with descriptions.}
%\item{Outputs(s) with descriptions.}
%\item{Function/ output summary (along with intermediate step comments).}
%\end{enumerate} 
%
%\noindent
%For Lab 7, please provide the following deliverable items:
%
%\begin{enumerate}
%\item{Please provide your solutions using Markdown as a .pdf with the following naming convention: LastName\textunderscore FirstName\textunderscore Solutions\textunderscore Lab7.pdf.}
%\item{Provide your .Rmd file (this \textbf{MUST} compile) for the lab using the following naming convention: LastName\textunderscore FirstName\textunderscore Solutions\textunderscore Lab7.Rmd}
%\end{enumerate}


\end{document}