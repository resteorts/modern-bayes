% Options for packages loaded elsewhere
\PassOptionsToPackage{unicode}{hyperref}
\PassOptionsToPackage{hyphens}{url}
%
\documentclass[
]{article}
\usepackage{lmodern}
\usepackage{amssymb,amsmath}
\usepackage{ifxetex,ifluatex}
\ifnum 0\ifxetex 1\fi\ifluatex 1\fi=0 % if pdftex
  \usepackage[T1]{fontenc}
  \usepackage[utf8]{inputenc}
  \usepackage{textcomp} % provide euro and other symbols
\else % if luatex or xetex
  \usepackage{unicode-math}
  \defaultfontfeatures{Scale=MatchLowercase}
  \defaultfontfeatures[\rmfamily]{Ligatures=TeX,Scale=1}
\fi
% Use upquote if available, for straight quotes in verbatim environments
\IfFileExists{upquote.sty}{\usepackage{upquote}}{}
\IfFileExists{microtype.sty}{% use microtype if available
  \usepackage[]{microtype}
  \UseMicrotypeSet[protrusion]{basicmath} % disable protrusion for tt fonts
}{}
\makeatletter
\@ifundefined{KOMAClassName}{% if non-KOMA class
  \IfFileExists{parskip.sty}{%
    \usepackage{parskip}
  }{% else
    \setlength{\parindent}{0pt}
    \setlength{\parskip}{6pt plus 2pt minus 1pt}}
}{% if KOMA class
  \KOMAoptions{parskip=half}}
\makeatother
\usepackage{xcolor}
\IfFileExists{xurl.sty}{\usepackage{xurl}}{} % add URL line breaks if available
\IfFileExists{bookmark.sty}{\usepackage{bookmark}}{\usepackage{hyperref}}
\hypersetup{
  pdftitle={Lab 3: Intro to Decision Theory},
  pdfauthor={Rebecca C. Steorts and Xingyu Yan},
  hidelinks,
  pdfcreator={LaTeX via pandoc}}
\urlstyle{same} % disable monospaced font for URLs
\usepackage[margin=1in]{geometry}
\usepackage{color}
\usepackage{fancyvrb}
\newcommand{\VerbBar}{|}
\newcommand{\VERB}{\Verb[commandchars=\\\{\}]}
\DefineVerbatimEnvironment{Highlighting}{Verbatim}{commandchars=\\\{\}}
% Add ',fontsize=\small' for more characters per line
\usepackage{framed}
\definecolor{shadecolor}{RGB}{248,248,248}
\newenvironment{Shaded}{\begin{snugshade}}{\end{snugshade}}
\newcommand{\AlertTok}[1]{\textcolor[rgb]{0.94,0.16,0.16}{#1}}
\newcommand{\AnnotationTok}[1]{\textcolor[rgb]{0.56,0.35,0.01}{\textbf{\textit{#1}}}}
\newcommand{\AttributeTok}[1]{\textcolor[rgb]{0.77,0.63,0.00}{#1}}
\newcommand{\BaseNTok}[1]{\textcolor[rgb]{0.00,0.00,0.81}{#1}}
\newcommand{\BuiltInTok}[1]{#1}
\newcommand{\CharTok}[1]{\textcolor[rgb]{0.31,0.60,0.02}{#1}}
\newcommand{\CommentTok}[1]{\textcolor[rgb]{0.56,0.35,0.01}{\textit{#1}}}
\newcommand{\CommentVarTok}[1]{\textcolor[rgb]{0.56,0.35,0.01}{\textbf{\textit{#1}}}}
\newcommand{\ConstantTok}[1]{\textcolor[rgb]{0.00,0.00,0.00}{#1}}
\newcommand{\ControlFlowTok}[1]{\textcolor[rgb]{0.13,0.29,0.53}{\textbf{#1}}}
\newcommand{\DataTypeTok}[1]{\textcolor[rgb]{0.13,0.29,0.53}{#1}}
\newcommand{\DecValTok}[1]{\textcolor[rgb]{0.00,0.00,0.81}{#1}}
\newcommand{\DocumentationTok}[1]{\textcolor[rgb]{0.56,0.35,0.01}{\textbf{\textit{#1}}}}
\newcommand{\ErrorTok}[1]{\textcolor[rgb]{0.64,0.00,0.00}{\textbf{#1}}}
\newcommand{\ExtensionTok}[1]{#1}
\newcommand{\FloatTok}[1]{\textcolor[rgb]{0.00,0.00,0.81}{#1}}
\newcommand{\FunctionTok}[1]{\textcolor[rgb]{0.00,0.00,0.00}{#1}}
\newcommand{\ImportTok}[1]{#1}
\newcommand{\InformationTok}[1]{\textcolor[rgb]{0.56,0.35,0.01}{\textbf{\textit{#1}}}}
\newcommand{\KeywordTok}[1]{\textcolor[rgb]{0.13,0.29,0.53}{\textbf{#1}}}
\newcommand{\NormalTok}[1]{#1}
\newcommand{\OperatorTok}[1]{\textcolor[rgb]{0.81,0.36,0.00}{\textbf{#1}}}
\newcommand{\OtherTok}[1]{\textcolor[rgb]{0.56,0.35,0.01}{#1}}
\newcommand{\PreprocessorTok}[1]{\textcolor[rgb]{0.56,0.35,0.01}{\textit{#1}}}
\newcommand{\RegionMarkerTok}[1]{#1}
\newcommand{\SpecialCharTok}[1]{\textcolor[rgb]{0.00,0.00,0.00}{#1}}
\newcommand{\SpecialStringTok}[1]{\textcolor[rgb]{0.31,0.60,0.02}{#1}}
\newcommand{\StringTok}[1]{\textcolor[rgb]{0.31,0.60,0.02}{#1}}
\newcommand{\VariableTok}[1]{\textcolor[rgb]{0.00,0.00,0.00}{#1}}
\newcommand{\VerbatimStringTok}[1]{\textcolor[rgb]{0.31,0.60,0.02}{#1}}
\newcommand{\WarningTok}[1]{\textcolor[rgb]{0.56,0.35,0.01}{\textbf{\textit{#1}}}}
\usepackage{graphicx,grffile}
\makeatletter
\def\maxwidth{\ifdim\Gin@nat@width>\linewidth\linewidth\else\Gin@nat@width\fi}
\def\maxheight{\ifdim\Gin@nat@height>\textheight\textheight\else\Gin@nat@height\fi}
\makeatother
% Scale images if necessary, so that they will not overflow the page
% margins by default, and it is still possible to overwrite the defaults
% using explicit options in \includegraphics[width, height, ...]{}
\setkeys{Gin}{width=\maxwidth,height=\maxheight,keepaspectratio}
% Set default figure placement to htbp
\makeatletter
\def\fps@figure{htbp}
\makeatother
\setlength{\emergencystretch}{3em} % prevent overfull lines
\providecommand{\tightlist}{%
  \setlength{\itemsep}{0pt}\setlength{\parskip}{0pt}}
\setcounter{secnumdepth}{-\maxdimen} % remove section numbering

\title{Lab 3: Intro to Decision Theory}
\author{Rebecca C. Steorts and Xingyu Yan}
\date{}

\begin{document}
\maketitle

In class, you saw the resource allocation example. We will now go
through how to reproduce parts of the lecture using R in Tasks 1-2 and
Tasks 3--5 should be completed in your weekly homework assignment. Let's
briefly recall the problem statement and set up.

Suppose public health officials in a small city need to decide how much
resources to devote toward prevention and treatment of a certain
disease, but the fraction \(\theta\) of infected individuals in the city
is unknown.

Suppose they allocate enough resources to accomodate a fraction \(c\) of
the population. If \(c\) is too large, there will be wasted resources,
while if it is too small, preventable cases may occur and some
individuals may go untreated. After deliberation, they tentatively adopt
the following loss function:
\[\ell(\theta,c) =\branch{|\theta-c|}{c\geq\theta}
                       {10|\theta-c|}{c<\theta.}\]

By considering data from other similar cities, they determine a prior
\(p(\theta)\). For simplicity, suppose \(\btheta\sim\Beta(a,b)\) (i.e.,
\(p(\theta) =\Beta(\theta|a,b)\)), with \(a=0.05\) and \(b=1\). They
conduct a survey assessing the disease status of \(n=30\) individuals,
\(x_1,\ldots,x_n\). This is modeled as
\(X_1,\ldots,X_n \stackrel{iid}{\sim} \Bernoulli(\theta)\), which is
reasonable if the individuals are uniformly sampled and the population
is large. Suppose all but one are disease-free, i.e.,
\(\sum_{i=1}^n x_i = 1\).

\hypertarget{task-1}{%
\section{Task 1}\label{task-1}}

\begin{Shaded}
\begin{Highlighting}[]
\CommentTok{# set seed }
\KeywordTok{set.seed}\NormalTok{(}\DecValTok{123}\NormalTok{)}

\CommentTok{# data}
\NormalTok{sum_x =}\StringTok{ }\DecValTok{1}
\NormalTok{n =}\StringTok{ }\DecValTok{30}
\CommentTok{# prior parameters}
\NormalTok{a =}\StringTok{ }\FloatTok{0.05}\NormalTok{; b =}\StringTok{ }\DecValTok{1}
\CommentTok{# posterior parameters}
\NormalTok{an =}\StringTok{ }\NormalTok{a }\OperatorTok{+}\StringTok{ }\NormalTok{sum_x}
\NormalTok{bn =}\StringTok{ }\NormalTok{b }\OperatorTok{+}\StringTok{ }\NormalTok{n }\OperatorTok{-}\StringTok{ }\NormalTok{sum_x}
\NormalTok{th =}\StringTok{ }\KeywordTok{seq}\NormalTok{(}\DecValTok{0}\NormalTok{,}\DecValTok{1}\NormalTok{,}\DataTypeTok{length.out =} \DecValTok{100}\NormalTok{)}
\NormalTok{like =}\StringTok{ }\KeywordTok{dbeta}\NormalTok{(th, sum_x}\OperatorTok{+}\DecValTok{1}\NormalTok{,n}\OperatorTok{-}\NormalTok{sum_x}\OperatorTok{+}\DecValTok{1}\NormalTok{)}
\NormalTok{prior =}\StringTok{ }\KeywordTok{dbeta}\NormalTok{(th,a,b)}
\NormalTok{post =}\StringTok{ }\KeywordTok{dbeta}\NormalTok{(th,sum_x}\OperatorTok{+}\NormalTok{a,n}\OperatorTok{-}\NormalTok{sum_x}\OperatorTok{+}\NormalTok{b)}
\end{Highlighting}
\end{Shaded}

We now consider the loss function.

\begin{Shaded}
\begin{Highlighting}[]
\CommentTok{# compute the loss given theta and c }
\NormalTok{loss_function =}\StringTok{ }\ControlFlowTok{function}\NormalTok{(theta, c)\{}
  \ControlFlowTok{if}\NormalTok{ (c }\OperatorTok{<}\StringTok{ }\NormalTok{theta)\{}
    \KeywordTok{return}\NormalTok{(}\DecValTok{10}\OperatorTok{*}\KeywordTok{abs}\NormalTok{(theta }\OperatorTok{-}\StringTok{ }\NormalTok{c))}
\NormalTok{  \} }\ControlFlowTok{else}\NormalTok{\{}
    \KeywordTok{return}\NormalTok{(}\DataTypeTok{l =} \KeywordTok{abs}\NormalTok{(theta }\OperatorTok{-}\StringTok{ }\NormalTok{c))}
\NormalTok{  \}}
\NormalTok{\}}
\end{Highlighting}
\end{Shaded}

We now write a function \textbf{posterior\_risk} which is a function of
c, parameters a\_prior and b\_prior for the prior distribution of
\(\theta\), the summation of \(x_i\) sum\_x, the number of observations
n, and also the number of random draws s.

\begin{Shaded}
\begin{Highlighting}[]
\CommentTok{# compute the posterior risk given c }
\CommentTok{# s is the number of random draws }
\NormalTok{posterior_risk =}\StringTok{ }\ControlFlowTok{function}\NormalTok{(c, a_prior, b_prior, sum_x, n, }\DataTypeTok{s =} \DecValTok{30000}\NormalTok{)\{}
  \CommentTok{# randow draws from beta distribution }
\NormalTok{  a_post =}\StringTok{ }\NormalTok{a_prior }\OperatorTok{+}\StringTok{ }\NormalTok{sum_x}
\NormalTok{  b_post =}\StringTok{ }\NormalTok{b_prior }\OperatorTok{+}\StringTok{ }\NormalTok{n }\OperatorTok{-}\StringTok{ }\NormalTok{sum_x}
\NormalTok{  theta =}\StringTok{ }\KeywordTok{rbeta}\NormalTok{(s, a_post, b_post)}
\NormalTok{  loss <-}\StringTok{ }\KeywordTok{apply}\NormalTok{(}\KeywordTok{as.matrix}\NormalTok{(theta),}\DecValTok{1}\NormalTok{,loss_function,c)}
  \CommentTok{# average values from the loss function}
\NormalTok{  risk =}\StringTok{ }\KeywordTok{mean}\NormalTok{(loss)}
\NormalTok{\}}
\CommentTok{# a sequence of c in [0, 0.5]}
\NormalTok{c =}\StringTok{ }\KeywordTok{seq}\NormalTok{(}\DecValTok{0}\NormalTok{, }\FloatTok{0.5}\NormalTok{, }\DataTypeTok{by =} \FloatTok{0.01}\NormalTok{)}
\NormalTok{post_risk <-}\StringTok{ }\KeywordTok{apply}\NormalTok{(}\KeywordTok{as.matrix}\NormalTok{(c),}\DecValTok{1}\NormalTok{,posterior_risk, a, b, sum_x, n)}
\KeywordTok{head}\NormalTok{(post_risk)}
\end{Highlighting}
\end{Shaded}

\begin{verbatim}
## [1] 0.33917940 0.25367603 0.18868962 0.14489894 0.11693106 0.09453471
\end{verbatim}

We then look at the Posterior expected loss (posterior risk) for disease
prevelance versus c.~

\begin{Shaded}
\begin{Highlighting}[]
\CommentTok{# plot posterior risk against c }
\KeywordTok{plot}\NormalTok{(c, post_risk, }\DataTypeTok{type =} \StringTok{'l'}\NormalTok{, }\DataTypeTok{col=}\StringTok{'blue'}\NormalTok{, }
    \DataTypeTok{lwd =} \DecValTok{3}\NormalTok{, }\DataTypeTok{ylab =}\StringTok{'p(c, x)'}\NormalTok{ )}
\end{Highlighting}
\end{Shaded}

\includegraphics{lab-03_files/figure-latex/unnamed-chunk-4-1.pdf}

\begin{Shaded}
\begin{Highlighting}[]
\CommentTok{# minimum of posterior risk occurs at c = 0.08}
\NormalTok{(c[}\KeywordTok{which.min}\NormalTok{(post_risk)])}
\end{Highlighting}
\end{Shaded}

\begin{verbatim}
## [1] 0.08
\end{verbatim}

\hypertarget{task-2}{%
\section{Task 2}\label{task-2}}

We now consider task 2. We set \(a = 0.05, 1, 0.05\) and
\(b = 1, 2, 10\). If we have different prior, the posterior risk is
minimized at different c values. The optimal c depends on not only the
data, but also the prior setting.

\begin{Shaded}
\begin{Highlighting}[]
\CommentTok{# set prior}
\NormalTok{as =}\StringTok{ }\KeywordTok{c}\NormalTok{(}\FloatTok{0.05}\NormalTok{, }\DecValTok{1}\NormalTok{, }\FloatTok{0.05}\NormalTok{); bs =}\StringTok{ }\KeywordTok{c}\NormalTok{(}\DecValTok{1}\NormalTok{, }\DecValTok{1}\NormalTok{, }\DecValTok{10}\NormalTok{)}
\NormalTok{post_risk =}\StringTok{ }\KeywordTok{matrix}\NormalTok{(}\OtherTok{NA}\NormalTok{, }\DecValTok{3}\NormalTok{, }\KeywordTok{length}\NormalTok{(c))}

\CommentTok{# for each pair of a and b, compute the posterior risks}
\ControlFlowTok{for}\NormalTok{ (i }\ControlFlowTok{in} \DecValTok{1}\OperatorTok{:}\DecValTok{3}\NormalTok{)\{}
\NormalTok{  a_prior =}\StringTok{ }\NormalTok{as[i]}
\NormalTok{  b_prior =}\StringTok{ }\NormalTok{bs[i]}
  
\NormalTok{  post_risk[i,] =}\StringTok{ }\KeywordTok{apply}\NormalTok{(}\KeywordTok{as.matrix}\NormalTok{(c), }\DecValTok{1}\NormalTok{, posterior_risk, a_prior, b_prior, sum_x, n)}
\NormalTok{\}}

\KeywordTok{plot}\NormalTok{(c, post_risk[}\DecValTok{1}\NormalTok{,], }\DataTypeTok{type =} \StringTok{'l'}\NormalTok{, }\DataTypeTok{col=}\StringTok{'blue'}\NormalTok{, }\DataTypeTok{lty =} \DecValTok{1}\NormalTok{, }\DataTypeTok{yaxt =} \StringTok{"n"}\NormalTok{, }\DataTypeTok{ylab =} \StringTok{"p(c, x)"}\NormalTok{)}
\KeywordTok{par}\NormalTok{(}\DataTypeTok{new =}\NormalTok{ T)}
\KeywordTok{plot}\NormalTok{(c, post_risk[}\DecValTok{2}\NormalTok{,], }\DataTypeTok{type =} \StringTok{'l'}\NormalTok{, }\DataTypeTok{col=}\StringTok{'red'}\NormalTok{, }\DataTypeTok{lty =} \DecValTok{2}\NormalTok{, }\DataTypeTok{yaxt =} \StringTok{"n"}\NormalTok{, }\DataTypeTok{ylab =} \StringTok{""}\NormalTok{)}
\KeywordTok{par}\NormalTok{(}\DataTypeTok{new =}\NormalTok{ T)}
\KeywordTok{plot}\NormalTok{(c, post_risk[}\DecValTok{3}\NormalTok{,], }\DataTypeTok{type =} \StringTok{'l'}\NormalTok{, }\DataTypeTok{lty =} \DecValTok{3}\NormalTok{, }\DataTypeTok{yaxt =} \StringTok{"n"}\NormalTok{, }\DataTypeTok{ylab =} \StringTok{""}\NormalTok{)}
\KeywordTok{legend}\NormalTok{(}\StringTok{"bottomright"}\NormalTok{, }\DataTypeTok{lty =} \KeywordTok{c}\NormalTok{(}\DecValTok{1}\NormalTok{,}\DecValTok{2}\NormalTok{,}\DecValTok{3}\NormalTok{), }\DataTypeTok{col =} \KeywordTok{c}\NormalTok{(}\StringTok{"blue"}\NormalTok{, }\StringTok{"red"}\NormalTok{, }\StringTok{"black"}\NormalTok{), }
       \DataTypeTok{legend =} \KeywordTok{c}\NormalTok{(}\StringTok{"a = 0.05 b = 1"}\NormalTok{, }\StringTok{"a = 1 b = 1"}\NormalTok{, }\StringTok{"a = 0.05 b = 5"}\NormalTok{))}
\end{Highlighting}
\end{Shaded}

\includegraphics{lab-03_files/figure-latex/unnamed-chunk-5-1.pdf}

Note there is a more automated solution but this is the most simple one
and is completely correct.

\hypertarget{task-3}{%
\section{Task 3}\label{task-3}}

The Bayes procedure always picks c to be a little bigger than
\(\bar{x}\).

\begin{Shaded}
\begin{Highlighting}[]
\NormalTok{sum_xs =}\StringTok{ }\KeywordTok{seq}\NormalTok{(}\DecValTok{0}\NormalTok{, }\DecValTok{30}\NormalTok{)}
\NormalTok{min_c =}\StringTok{ }\KeywordTok{matrix}\NormalTok{(}\OtherTok{NA}\NormalTok{, }\DecValTok{3}\NormalTok{, }\KeywordTok{length}\NormalTok{(sum_xs))}

\CommentTok{# find_optimal_C finds the optimal c under Bayes procedure}
\CommentTok{# function of sum of x, parameters for prior, number of observations, and number of random draws }
\NormalTok{find_optimal_C =}\StringTok{ }\ControlFlowTok{function}\NormalTok{(sum_x, a_prior, b_prior, n, }\DataTypeTok{s =} \DecValTok{500}\NormalTok{)\{}
\NormalTok{  c =}\StringTok{ }\KeywordTok{seq}\NormalTok{(}\DecValTok{0}\NormalTok{, }\DecValTok{1}\NormalTok{, }\DataTypeTok{by =} \FloatTok{0.01}\NormalTok{)}
\NormalTok{  post_risk =}\StringTok{  }\KeywordTok{apply}\NormalTok{(}\KeywordTok{as.matrix}\NormalTok{(c), }\DecValTok{1}\NormalTok{, posterior_risk, a_prior, b_prior, sum_x, n, s)}
\NormalTok{  c[}\KeywordTok{which.min}\NormalTok{(post_risk)]}
\NormalTok{\}}

\NormalTok{min_c[}\DecValTok{1}\NormalTok{,] =}\StringTok{ }\KeywordTok{apply}\NormalTok{(}\KeywordTok{as.matrix}\NormalTok{(sum_xs), }\DecValTok{1}\NormalTok{, find_optimal_C, a, b, n)}
\CommentTok{# find optimal c under sample mean}
\NormalTok{min_c[}\DecValTok{2}\NormalTok{,] =}\StringTok{ }\NormalTok{(sum_xs)}\OperatorTok{/}\NormalTok{n}
\CommentTok{# constant c }
\NormalTok{min_c[}\DecValTok{3}\NormalTok{,] =}\StringTok{ }\FloatTok{0.1}

\CommentTok{# plot}
\KeywordTok{plot}\NormalTok{(sum_xs, min_c[}\DecValTok{1}\NormalTok{,], }\DataTypeTok{col=}\StringTok{'blue'}\NormalTok{,}\DataTypeTok{type =} \StringTok{'o'}\NormalTok{,}\DataTypeTok{pch =} \DecValTok{16}\NormalTok{, }
     \DataTypeTok{ylab =} \StringTok{"resources allocated"}\NormalTok{, }\DataTypeTok{xlab =} \StringTok{'observed number of diseased cases'}\NormalTok{,}
     \DataTypeTok{ylim =} \KeywordTok{c}\NormalTok{(}\DecValTok{0}\NormalTok{,}\DecValTok{1}\NormalTok{))}
\KeywordTok{par}\NormalTok{(}\DataTypeTok{new =}\NormalTok{ T)}
\KeywordTok{plot}\NormalTok{(sum_xs, min_c[}\DecValTok{2}\NormalTok{,], }\DataTypeTok{type =} \StringTok{'o'}\NormalTok{, }\DataTypeTok{col=}\StringTok{'green'}\NormalTok{, }
     \DataTypeTok{pch =} \DecValTok{16}\NormalTok{, }\DataTypeTok{ylab =} \StringTok{""}\NormalTok{, }\DataTypeTok{xlab =} \StringTok{''}\NormalTok{, }\DataTypeTok{ylim =} \KeywordTok{c}\NormalTok{(}\DecValTok{0}\NormalTok{,}\DecValTok{1}\NormalTok{))}
\KeywordTok{par}\NormalTok{(}\DataTypeTok{new =}\NormalTok{ T)}
\KeywordTok{plot}\NormalTok{(sum_xs, min_c[}\DecValTok{3}\NormalTok{,], }\DataTypeTok{type =} \StringTok{'o'}\NormalTok{,}\DataTypeTok{col =} \StringTok{'red'}\NormalTok{,}
     \DataTypeTok{pch =} \DecValTok{16}\NormalTok{, }\DataTypeTok{ylab =} \StringTok{""}\NormalTok{, }\DataTypeTok{xlab =} \StringTok{''}\NormalTok{, }\DataTypeTok{ylim =} \KeywordTok{c}\NormalTok{(}\DecValTok{0}\NormalTok{,}\DecValTok{1}\NormalTok{))}
\KeywordTok{legend}\NormalTok{(}\StringTok{"topleft"}\NormalTok{, }\DataTypeTok{lty =} \KeywordTok{c}\NormalTok{(}\DecValTok{1}\NormalTok{,}\DecValTok{1}\NormalTok{,}\DecValTok{1}\NormalTok{), }\DataTypeTok{pch =} \KeywordTok{c}\NormalTok{(}\DecValTok{16}\NormalTok{,}\DecValTok{16}\NormalTok{,}\DecValTok{16}\NormalTok{),}
       \DataTypeTok{col =} \KeywordTok{c}\NormalTok{(}\StringTok{"blue"}\NormalTok{, }\StringTok{"green"}\NormalTok{, }\StringTok{"red"}\NormalTok{),}
       \DataTypeTok{legend =} \KeywordTok{c}\NormalTok{(}\StringTok{"Bayes"}\NormalTok{, }\StringTok{"Sample mean"}\NormalTok{, }\StringTok{"constant"}\NormalTok{))}
\end{Highlighting}
\end{Shaded}

\includegraphics{lab-03_files/figure-latex/unnamed-chunk-6-1.pdf}

\hypertarget{task-4}{%
\section{Task 4}\label{task-4}}

For all \(\theta\), the Bayes procedure has the lower frequentist risk
than the sample mean.

\begin{Shaded}
\begin{Highlighting}[]
\NormalTok{thetas =}\StringTok{ }\KeywordTok{seq}\NormalTok{(}\DecValTok{0}\NormalTok{, }\DecValTok{1}\NormalTok{, }\DataTypeTok{by=}\FloatTok{0.1}\NormalTok{)}

\CommentTok{# frequentist risk for the 3 estimators given a theta}
\NormalTok{frequentist_risk =}\StringTok{ }\ControlFlowTok{function}\NormalTok{(theta)\{}
\NormalTok{  sum_xs =}\StringTok{ }\KeywordTok{rbinom}\NormalTok{(}\DecValTok{100}\NormalTok{, }\DecValTok{30}\NormalTok{, theta)}
\NormalTok{  Bayes_optimal =}\StringTok{ }\KeywordTok{apply}\NormalTok{(}\KeywordTok{as.matrix}\NormalTok{(sum_xs), }\DecValTok{1}\NormalTok{, find_optimal_C, a, b, n, }\DataTypeTok{s =} \DecValTok{100}\NormalTok{)}
\NormalTok{  mean_c =}\StringTok{ }\NormalTok{sum_xs }\OperatorTok{/}\StringTok{ }\DecValTok{30}
  
\NormalTok{  loss1 =}\StringTok{ }\KeywordTok{apply}\NormalTok{(}\KeywordTok{as.matrix}\NormalTok{(Bayes_optimal), }\DecValTok{1}\NormalTok{, loss_function, }\DataTypeTok{theta =}\NormalTok{ theta)}
\NormalTok{  loss2 =}\StringTok{ }\KeywordTok{apply}\NormalTok{(}\KeywordTok{as.matrix}\NormalTok{(mean_c), }\DecValTok{1}\NormalTok{, loss_function, }\DataTypeTok{theta =}\NormalTok{ theta )}
\NormalTok{  risk1 =}\StringTok{ }\KeywordTok{mean}\NormalTok{(loss1)}
\NormalTok{  risk2 =}\StringTok{ }\KeywordTok{mean}\NormalTok{(loss2)}
\NormalTok{  risk3 =}\StringTok{ }\KeywordTok{loss_function}\NormalTok{(theta, }\FloatTok{0.1}\NormalTok{)}
  \KeywordTok{return}\NormalTok{(}\KeywordTok{c}\NormalTok{(risk1, risk2, risk3))}
\NormalTok{\}}

\CommentTok{# given a sequance a theta, compute frequentist risk for each theta}
\NormalTok{R =}\StringTok{ }\KeywordTok{apply}\NormalTok{(}\KeywordTok{as.matrix}\NormalTok{(thetas), }\DecValTok{1}\NormalTok{, frequentist_risk)}

\CommentTok{# plot}
\KeywordTok{plot}\NormalTok{(thetas, R[}\DecValTok{1}\NormalTok{,], }\DataTypeTok{col=}\StringTok{'blue'}\NormalTok{, }\DataTypeTok{type =} \StringTok{"l"}\NormalTok{, }
     \DataTypeTok{ylab =} \StringTok{"frequentist risk"}\NormalTok{, }\DataTypeTok{xlab =} \StringTok{'theta'}\NormalTok{,}\DataTypeTok{ylim =} \KeywordTok{c}\NormalTok{(}\DecValTok{0}\NormalTok{,}\DecValTok{1}\NormalTok{))}
\KeywordTok{par}\NormalTok{(}\DataTypeTok{new =}\NormalTok{ T)}
\KeywordTok{plot}\NormalTok{(thetas, R[}\DecValTok{2}\NormalTok{,], }\DataTypeTok{type =} \StringTok{'l'}\NormalTok{, }\DataTypeTok{col=}\StringTok{'green'}\NormalTok{, }
     \DataTypeTok{ylab =} \StringTok{""}\NormalTok{, }\DataTypeTok{xlab =} \StringTok{''}\NormalTok{, }\DataTypeTok{ylim =} \KeywordTok{c}\NormalTok{(}\DecValTok{0}\NormalTok{,}\DecValTok{1}\NormalTok{))}
\KeywordTok{par}\NormalTok{(}\DataTypeTok{new =}\NormalTok{ T)}
\KeywordTok{plot}\NormalTok{(thetas, R[}\DecValTok{3}\NormalTok{,], }\DataTypeTok{type =} \StringTok{'l'}\NormalTok{,}\DataTypeTok{col =} \StringTok{'red'}\NormalTok{,}
     \DataTypeTok{ylab =} \StringTok{""}\NormalTok{, }\DataTypeTok{xlab =} \StringTok{''}\NormalTok{, }\DataTypeTok{ylim =} \KeywordTok{c}\NormalTok{(}\DecValTok{0}\NormalTok{,}\DecValTok{1}\NormalTok{))}
\KeywordTok{legend}\NormalTok{(}\StringTok{"topright"}\NormalTok{, }\DataTypeTok{lty =} \KeywordTok{c}\NormalTok{(}\DecValTok{1}\NormalTok{,}\DecValTok{1}\NormalTok{,}\DecValTok{1}\NormalTok{), }\DataTypeTok{col =} \KeywordTok{c}\NormalTok{(}\StringTok{"blue"}\NormalTok{, }\StringTok{"green"}\NormalTok{, }\StringTok{"red"}\NormalTok{),}
       \DataTypeTok{legend =} \KeywordTok{c}\NormalTok{(}\StringTok{"Bayes"}\NormalTok{, }\StringTok{"Sample mean"}\NormalTok{, }\StringTok{"constant"}\NormalTok{))}
\end{Highlighting}
\end{Shaded}

\includegraphics{lab-03_files/figure-latex/unnamed-chunk-7-1.pdf}

Please see a few remarks about Task 4 that will help you with
interpreting the plot.

\begin{enumerate}
\item If you zoom into see the plot for Task 4, you will notice that the Bayes risk is not always smaller than the sample mean. Specifically, the issue is occuring around $\theta = 0$ and $\theta = 1.$
\item One observation that we can make is that when $x$ is very small (say 0), Bayes estimator tends to overestimate $\theta$ and hence sample mean has lower risk. What other observations can you make? 
\end{enumerate}

I am including some code that Xu Chen has written that is much faster,
where the resulting plot is much more clear.

\begin{Shaded}
\begin{Highlighting}[]
\CommentTok{# code by Xu Chen}
\NormalTok{loss <-}\StringTok{ }\ControlFlowTok{function}\NormalTok{(theta, c)\{}
  \ControlFlowTok{if}\NormalTok{ (c }\OperatorTok{>=}\StringTok{ }\NormalTok{theta) \{}
    \KeywordTok{return}\NormalTok{(c }\OperatorTok{-}\StringTok{ }\NormalTok{theta)}
\NormalTok{  \} }\ControlFlowTok{else}\NormalTok{ \{}
    \KeywordTok{return}\NormalTok{(}\DecValTok{10}\OperatorTok{*}\NormalTok{(theta }\OperatorTok{-}\StringTok{ }\NormalTok{c))}
\NormalTok{  \}}
\NormalTok{\}}


\NormalTok{delta1 <-}\StringTok{ }\ControlFlowTok{function}\NormalTok{(x)\{}
  \KeywordTok{return}\NormalTok{(}\KeywordTok{rep}\NormalTok{(}\FloatTok{0.1}\NormalTok{,}\KeywordTok{length}\NormalTok{(x)))}
\NormalTok{\}}


\NormalTok{delta2 <-}\StringTok{ }\ControlFlowTok{function}\NormalTok{(x)\{}
  \KeywordTok{return}\NormalTok{(x}\OperatorTok{/}\DecValTok{30}\NormalTok{)}
\NormalTok{\}}


\NormalTok{delta3 <-}\StringTok{ }\ControlFlowTok{function}\NormalTok{(x, }\DataTypeTok{a =} \FloatTok{0.05}\NormalTok{, }\DataTypeTok{b =} \DecValTok{1}\NormalTok{)\{}
\NormalTok{  a.post <-}\StringTok{ }\NormalTok{a }\OperatorTok{+}\StringTok{ }\NormalTok{x}
\NormalTok{  b.post <-}\StringTok{ }\NormalTok{b }\OperatorTok{+}\StringTok{ }\DecValTok{30} \OperatorTok{-}\StringTok{ }\NormalTok{x}
\NormalTok{  c <-}\StringTok{ }\KeywordTok{seq}\NormalTok{(}\DecValTok{0}\NormalTok{,}\DecValTok{1}\NormalTok{,}\FloatTok{0.01}\NormalTok{)}
\NormalTok{  theta <-}\StringTok{ }\KeywordTok{matrix}\NormalTok{(}\KeywordTok{rbeta}\NormalTok{(}\FloatTok{1e4}\OperatorTok{*}\KeywordTok{length}\NormalTok{(c), a.post, b.post), }\DataTypeTok{nrow =} \FloatTok{1e4}\NormalTok{, }\DataTypeTok{ncol =} \KeywordTok{length}\NormalTok{(c))}
  
  \KeywordTok{return}\NormalTok{(c[}\KeywordTok{which.min}\NormalTok{(}\KeywordTok{sapply}\NormalTok{(c, }\ControlFlowTok{function}\NormalTok{(u) }\KeywordTok{mean}\NormalTok{(}\KeywordTok{sapply}\NormalTok{(theta[,}\KeywordTok{as.integer}\NormalTok{(}\DecValTok{100}\OperatorTok{*}\NormalTok{u}\OperatorTok{+}\DecValTok{1}\NormalTok{)], loss, }\DataTypeTok{c =}\NormalTok{ u))))])}
\NormalTok{\}}

\NormalTok{risk <-}\StringTok{ }\ControlFlowTok{function}\NormalTok{(theta, c)\{}
  \KeywordTok{return}\NormalTok{(}\KeywordTok{sum}\NormalTok{(}\KeywordTok{dbinom}\NormalTok{(}\DataTypeTok{x =} \DecValTok{0}\OperatorTok{:}\DecValTok{30}\NormalTok{, }\DataTypeTok{size =} \DecValTok{30}\NormalTok{, }\DataTypeTok{prob =}\NormalTok{ theta) }\OperatorTok{*}\StringTok{ }\KeywordTok{sapply}\NormalTok{(c, loss, }\DataTypeTok{theta =}\NormalTok{ theta)))}
\NormalTok{\}}


\NormalTok{theta.grid <-}\StringTok{ }\KeywordTok{seq}\NormalTok{(}\DecValTok{0}\NormalTok{,}\DecValTok{1}\NormalTok{,}\FloatTok{0.01}\NormalTok{)}
\NormalTok{x <-}\StringTok{ }\DecValTok{0}\OperatorTok{:}\DecValTok{30}

\NormalTok{c3 <-}\StringTok{ }\KeywordTok{sapply}\NormalTok{(x, delta3)}
\KeywordTok{plot}\NormalTok{(theta.grid, }\KeywordTok{sapply}\NormalTok{(theta.grid, risk, c3), }\DataTypeTok{ylim =} \KeywordTok{c}\NormalTok{(}\DecValTok{0}\NormalTok{,}\DecValTok{1}\NormalTok{), }\DataTypeTok{type =} \StringTok{'l'}\NormalTok{, }\DataTypeTok{col =} \StringTok{'red'}\NormalTok{, }\DataTypeTok{xlab =} \KeywordTok{expression}\NormalTok{(theta), }\DataTypeTok{ylab =} \StringTok{'risk'}\NormalTok{)}

\NormalTok{c1 <-}\StringTok{ }\KeywordTok{delta1}\NormalTok{(x)}
\KeywordTok{points}\NormalTok{(theta.grid, }\KeywordTok{sapply}\NormalTok{(theta.grid, risk, c1), }\DataTypeTok{ylim =} \KeywordTok{c}\NormalTok{(}\DecValTok{0}\NormalTok{,}\DecValTok{1}\NormalTok{), }\DataTypeTok{type =} \StringTok{'l'}\NormalTok{)}

\NormalTok{c2 <-}\StringTok{ }\KeywordTok{delta2}\NormalTok{(x)}
\KeywordTok{points}\NormalTok{(theta.grid, }\KeywordTok{sapply}\NormalTok{(theta.grid, risk, c2), }\DataTypeTok{ylim =} \KeywordTok{c}\NormalTok{(}\DecValTok{0}\NormalTok{,}\DecValTok{1}\NormalTok{), }\DataTypeTok{type =} \StringTok{'l'}\NormalTok{, }\DataTypeTok{col=}\StringTok{'green'}\NormalTok{)}

\KeywordTok{legend}\NormalTok{(}\StringTok{'topright'}\NormalTok{, }\DataTypeTok{legend =} \KeywordTok{c}\NormalTok{(}\StringTok{'Bayes'}\NormalTok{, }\StringTok{'sample mean'}\NormalTok{, }\StringTok{'constant'}\NormalTok{), }\DataTypeTok{col =} \KeywordTok{c}\NormalTok{(}\StringTok{'red'}\NormalTok{, }\StringTok{'green'}\NormalTok{, }\StringTok{'black'}\NormalTok{), }\DataTypeTok{lty =} \DecValTok{1}\NormalTok{)}
\end{Highlighting}
\end{Shaded}

\includegraphics{lab-03_files/figure-latex/unnamed-chunk-8-1.pdf}

\end{document}
