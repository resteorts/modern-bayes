\documentclass[mathserif]{beamer}

\setbeamertemplate{frametitle}[default][center]%Centers the frame title.
\setbeamertemplate{navigation symbols}{}%Removes navigation symbols.
\setbeamertemplate{footline}{\raisebox{5pt}{\makebox[\paperwidth]{\hfill\makebox[10pt]{\scriptsize\insertframenumber}}}}
\setbeamertemplate{caption}[numbered]

\usepackage{amssymb,amsfonts,amsmath,latexsym,amsthm}
%\usepackage[usenames,dvipsnames]{color}
%\usepackage[]{graphicx}
%\usepackage[space]{grffile}
\usepackage{mathrsfs}   % fancy math font
% \usepackage[font=small,skip=0pt]{caption}
\usepackage[skip=0pt]{caption}
\usepackage{subcaption}
\usepackage{verbatim}
\usepackage{url}
\usepackage{bm}
\usepackage{dsfont}
\usepackage{extarrows}
\usepackage{multirow}
%\newcommand{\tth}   {\mbox{$\theta$}}
\newcommand{\thh}   {\mbox{$\theta$}}
\newcommand{\su}   {\mbox{$\sigma^2$}}
\newcommand{\so}   {\mbox{$\sigma_0^2$}}
\newcommand{\ko}   {\mbox{$\kappa_0$}}
\newcommand{\no}   {\mbox{$\nu_0$}}
\newcommand{\mo}   {\mbox{$\mu_0$}}
\newcommand{\ti}   {\mbox{$\tilde{x}$}}
\newcommand{\la}   {\mbox{$\lambda$}}
\newcommand{\bx}   {\mbox{$\bm{x}$}}
\newcommand{\bZ}   {\mbox{$\bm{Z}$}}
\newcommand{\bX}   {\mbox{$\bm{X}$}}
\newcommand{\bY}   {\mbox{$\bm{Y}$}}
\newcommand{\bA}   {\mbox{$\bm{A}$}}
\newcommand{\ba}   {\mbox{$\bm{a}$}}
\newcommand{\bb}   {\mbox{$\bm{b}$}}
\newcommand{\bt}   {\mbox{$\bm{t}$}}
\newcommand{\bz}   {\mbox{$\bm{z}$}}
\newcommand{\bw}   {\mbox{$\bm{w}$}}
\newcommand{\bbeta}   {\mbox{$\bm{\beta}$}}

\newcommand{\be}   {\mbox{$\bm{e}$}}
\newcommand{\bu}   {\mbox{$\bm{u}$}}
\newcommand{\bv}   {\mbox{$\bm{v}$}}
\newcommand{\sig}   {\mbox{$\Sigma$}}
\newcommand{\sigx}   {\mbox{$\Sigma_{XX}$}}
\newcommand{\sigxy}   {\mbox{$\Sigma_{XY}$}}
\newcommand{\tr}   {\mbox{$\text{tr}$}}
\newcommand{\ddet}   {\mbox{$\text{det}$}}
\newcommand\independent{\protect\mathpalette{\protect\independenT}{\perp}}
\def\independenT#1#2{\mathrel{\rlap{$#1#2$}\mkern2mu{#1#2}}}

\newcommand{\Expect}[1]{\ensuremath{\mathbf{E}\left[ #1 \right]}}
%\newcommand{\Var}[1]{\ensuremath{\mathrm{Var}\left[ #1 \right]}}
%\newcommand{\Cov}[1]{\ensuremath{\mathrm{Cov}\left[ #1 \right]}}
\newcommand{\MSE}{\ensuremath{\mathrm{MSE}}}
\newcommand{\RSS}{\ensuremath{\mathrm{RSS}}}
\newcommand{\Prob}[1]{\ensuremath{\mathrm{Pr}\left( #1 \right)}}
\newcommand{\ProbEst}[1]{\ensuremath{\widehat{\mathrm{Pr}}\left( #1 \right)}}
\DeclareMathOperator*{\argmin}{argmin} % thanks, wikipedia!
\DeclareMathOperator*{\argmax}{argmax} % thanks, wikipedia!
\DeclareMathOperator*{\sgn}{sgn} % thanks, wikipedia!

\newcommand{\lam}{\lambda}
\newcommand{\bmu}{\bm{\mu}}
%\newcommand{\bx}{\ensuremath{\mathbf{X}}}
\newcommand{\X}{\ensuremath{\mathbf{X}}}
\newcommand{\w}{\ensuremath{\mathbf{w}}}
\newcommand{\h}{\ensuremath{\mathbf{h}}}
\newcommand{\V}{\ensuremath{\mathbf{V}}}
%\newcommand{\tr}{\operatorname{tr}}

%\newcommand{\bx}{\ensuremath{\mathbf{X}}}
%\newcommand{\X}{\ensuremath{\mathbf{x}}}
%\newcommand{\w}{\ensuremath{\mathbf{w}}}
%\newcommand{\h}{\ensuremath{\mathbf{h}}}
%\newcommand{\V}{\ensuremath{\mathbf{v}}}
%\newcommand{\Cov}{\text{Cov}}
%\newcommand{\Var}{\text{Var}}

\DeclareMathOperator{\var}{Var}
\DeclareMathOperator{\cov}{Cov}
\newcommand{\Var}[1]{\ensuremath{\mathrm{Var}\left[ #1 \right]}}
\newcommand{\Cov}[1]{\ensuremath{\mathrm{Cov}\left[ #1 \right]}}


\newcommand{\indep}{\rotatebox{90}{\ensuremath{\models}}}
\newcommand{\notindep}{\not\hspace{-.05in}\indep}







\usepackage{float,bm}
\floatstyle{boxed}
\newfloat{code}{tp}{code}
\floatname{code}{Code Example}
%\newcommand{\tth}   {\mbox{$\theta$}}
\newcommand{\thh}   {\mbox{$\theta$}}
\newcommand{\su}   {\mbox{$\sigma^2$}}
\newcommand{\so}   {\mbox{$\sigma_0^2$}}
\newcommand{\ko}   {\mbox{$\kappa_0$}}
\newcommand{\no}   {\mbox{$\nu_0$}}
\newcommand{\mo}   {\mbox{$\mu_0$}}
\newcommand{\ti}   {\mbox{$\tilde{x}$}}
\newcommand{\la}   {\mbox{$\lambda$}}
\newcommand{\bx}   {\mbox{$\bm{x}$}}
\newcommand{\bZ}   {\mbox{$\bm{Z}$}}
\newcommand{\bX}   {\mbox{$\bm{X}$}}
\newcommand{\bY}   {\mbox{$\bm{Y}$}}
\newcommand{\bA}   {\mbox{$\bm{A}$}}
\newcommand{\ba}   {\mbox{$\bm{a}$}}
\newcommand{\bb}   {\mbox{$\bm{b}$}}
\newcommand{\bt}   {\mbox{$\bm{t}$}}
\newcommand{\bz}   {\mbox{$\bm{z}$}}
\newcommand{\bw}   {\mbox{$\bm{w}$}}
\newcommand{\bbeta}   {\mbox{$\bm{\beta}$}}

\newcommand{\be}   {\mbox{$\bm{e}$}}
\newcommand{\bu}   {\mbox{$\bm{u}$}}
\newcommand{\bv}   {\mbox{$\bm{v}$}}
\newcommand{\sig}   {\mbox{$\Sigma$}}
\newcommand{\sigx}   {\mbox{$\Sigma_{XX}$}}
\newcommand{\sigxy}   {\mbox{$\Sigma_{XY}$}}
\newcommand{\tr}   {\mbox{$\text{tr}$}}
\newcommand{\ddet}   {\mbox{$\text{det}$}}
\newcommand\independent{\protect\mathpalette{\protect\independenT}{\perp}}
\def\independenT#1#2{\mathrel{\rlap{$#1#2$}\mkern2mu{#1#2}}}

\newcommand{\Expect}[1]{\ensuremath{\mathbf{E}\left[ #1 \right]}}
%\newcommand{\Var}[1]{\ensuremath{\mathrm{Var}\left[ #1 \right]}}
%\newcommand{\Cov}[1]{\ensuremath{\mathrm{Cov}\left[ #1 \right]}}
\newcommand{\MSE}{\ensuremath{\mathrm{MSE}}}
\newcommand{\RSS}{\ensuremath{\mathrm{RSS}}}
\newcommand{\Prob}[1]{\ensuremath{\mathrm{Pr}\left( #1 \right)}}
\newcommand{\ProbEst}[1]{\ensuremath{\widehat{\mathrm{Pr}}\left( #1 \right)}}
\DeclareMathOperator*{\argmin}{argmin} % thanks, wikipedia!
\DeclareMathOperator*{\argmax}{argmax} % thanks, wikipedia!
\DeclareMathOperator*{\sgn}{sgn} % thanks, wikipedia!

\newcommand{\lam}{\lambda}
\newcommand{\bmu}{\bm{\mu}}
%\newcommand{\bx}{\ensuremath{\mathbf{X}}}
\newcommand{\X}{\ensuremath{\mathbf{X}}}
\newcommand{\w}{\ensuremath{\mathbf{w}}}
\newcommand{\h}{\ensuremath{\mathbf{h}}}
\newcommand{\V}{\ensuremath{\mathbf{V}}}
%\newcommand{\tr}{\operatorname{tr}}

%\newcommand{\bx}{\ensuremath{\mathbf{X}}}
%\newcommand{\X}{\ensuremath{\mathbf{x}}}
%\newcommand{\w}{\ensuremath{\mathbf{w}}}
%\newcommand{\h}{\ensuremath{\mathbf{h}}}
%\newcommand{\V}{\ensuremath{\mathbf{v}}}
%\newcommand{\Cov}{\text{Cov}}
%\newcommand{\Var}{\text{Var}}

\DeclareMathOperator{\var}{Var}
\DeclareMathOperator{\cov}{Cov}
\newcommand{\Var}[1]{\ensuremath{\mathrm{Var}\left[ #1 \right]}}
\newcommand{\Cov}[1]{\ensuremath{\mathrm{Cov}\left[ #1 \right]}}


\newcommand{\indep}{\rotatebox{90}{\ensuremath{\models}}}
\newcommand{\notindep}{\not\hspace{-.05in}\indep}






%\usepackage{fontspec}
%\setmainfont{Tahoma}

%\newcommand{\lam}{\lambda}
%\newcommand{\bmu}{\bm{\mu}}
%%\newcommand{\bx}{\ensuremath{\mathbf{X}}}
%\newcommand{\X}{\ensuremath{\mathbf{x}}}
%\newcommand{\w}{\ensuremath{\mathbf{w}}}
%\newcommand{\h}{\ensuremath{\mathbf{h}}}
%\newcommand{\V}{\ensuremath{\mathbf{v}}}
%\newcommand{\cov}{\text{Cov}}
%\newcommand{\var{\text{Var}}}

%\DeclareMathOperator{\var}{Var}
%\DeclareMathOperator{\cov}{Cov}

%\newcommand{\indep}{\rotatebox{90}{\ensuremath{\models}}}
%\newcommand{\notindep}{\not\hspace{-.05in}\indep}

\usepackage{graphicx} %The mode "LaTeX => PDF" allows the following formats: .jpg  .png  .pdf  .mps
\graphicspath{{./PresentationPictures/}} %Where the figures folder is located
\usepackage{listings}
\usepackage{media9}
\usepackage{movie15}
\addmediapath{./Movies/}

\newcommand{\beginbackup}{
   \newcounter{framenumbervorappendix}
   \setcounter{framenumbervorappendix}{\value{framenumber}}
}
\newcommand{\backupend}{
   \addtocounter{framenumbervorappendix}{-\value{framenumber}}
   \addtocounter{framenumber}{\value{framenumbervorappendix}} 
}


%\usepackage{algorithm2e}
\usepackage[ruled,lined]{algorithm2e}
\def\algorithmautorefname{Algorithm}
\SetKwIF{If}{ElseIf}{Else}{if}{then}{else if}{else}{endif}
%\usepackage{times}
%\usepackage[tbtags]{amsmath}
%\usepackage{amssymb}
\usepackage{amsfonts}
%\usepackage{slfortheorems}
\usepackage{epsfig}
\usepackage{graphicx}
%\usepackage[small]{caption}
%\usepackage[square]{natbib}
%\newcommand{\newblock}{}
%\bibpunct{(}{)}{;}{a}{}{,}
%\bibliographystyle{ims}
%\usepackage[letterpaper]{geometry}
%\usepackage{color}
%\setlength{\parindent}{0pt}

\usepackage{natbib}
\bibpunct{(}{)}{;}{a}{}{,}
%\usepackage{hyperref}

\DeclareMathOperator*{\Exp}{Exp}
\DeclareMathOperator*{\TExp}{TExp}
\DeclareMathOperator*{\Bernoulli}{Bernoulli}
\DeclareMathOperator*{\Beta}{Beta}
\DeclareMathOperator*{\Ga}{Gamma}
\DeclareMathOperator*{\TGamma}{TGamma}
\DeclareMathOperator*{\Poisson}{Poisson}
\DeclareMathOperator*{\Binomial}{Binomial}
\DeclareMathOperator*{\NormalGamma}{NormalGamma}
\DeclareMathOperator*{\InvGamma}{InvGamma}
\DeclareMathOperator*{\Cauchy}{Cauchy}
\DeclareMathOperator*{\Uniform}{Uniform}
\DeclareMathOperator*{\Gumbel}{Gumbel}
\DeclareMathOperator*{\Pareto}{Pareto}
\DeclareMathOperator*{\Mono}{Mono}
\DeclareMathOperator*{\Geometric}{Geometric}
\DeclareMathOperator*{\Wishart}{Wishart}

\newcommand{\N}{\mathcal{N}}

\newcommand{\R}{\mathbb{R}}
\newcommand{\Z}{\mathbb{Z}}
\newcommand{\E}{\mathbb{E}}
\renewcommand{\Pr}{\mathbb{P}}
\newcommand{\I}{\mathds{1}}
\newcommand{\V}{\mathbb{V}}

% Math operators
\DeclareMathOperator*{\diag}{diag}
\DeclareMathOperator*{\median}{median}
\DeclareMathOperator*{\Vol}{Vol}

% Miscellaneous commands
\newcommand{\iid}{\stackrel{\mathrm{iid}}{\sim}}
\newcommand{\matrixsmall}[1]{\bigl(\begin{smallmatrix}#1\end{smallmatrix} \bigr)}

\newcommand{\items}[1]{\begin{itemize} #1 \end{itemize}}

\newcommand{\todo}[1]{\emph{\textcolor{red}{(#1)}}}

\newcommand{\branch}[4]{
\left\{
	\begin{array}{ll}
		#1  & \mbox{if } #2 \\
		#3 & \mbox{if } #4
	\end{array}
\right.
}

% approximately proportional to
\def\app#1#2{%
  \mathrel{%
    \setbox0=\hbox{$#1\sim$}%
    \setbox2=\hbox{%
      \rlap{\hbox{$#1\propto$}}%
      \lower1.3\ht0\box0%
    }%
    \raise0.25\ht2\box2%
  }%
}
\def\approxprop{\mathpalette\app\relax}

\newcommand{\btheta}{{\bm\theta}}
\newcommand{\bbtheta}{{\pmb{\bm\theta}}}

%\usepackage{zref-savepos}
%
%\newcounter{restofframe}
%\newsavebox{\restofframebox}
%\newlength{\mylowermargin}
%\setlength{\mylowermargin}{2pt}
%
%\newenvironment{restofframe}{%
%    \par%\centering
%    \stepcounter{restofframe}%
%    \zsavepos{restofframe-\arabic{restofframe}-begin}%
%    \begin{lrbox}{\restofframebox}%
%}{%
%    \end{lrbox}%
%    \setkeys{Gin}{keepaspectratio}%
%    \raisebox{\dimexpr-\height+\ht\strutbox\relax}[0pt][0pt]{%
%    \resizebox*{!}{\dimexpr\zposy{restofframe-\arabic{restofframe}-begin}sp-\zposy{restofframe-\arabic{restofframe}-end}sp-\mylowermargin\relax}%
%        {\usebox{\restofframebox}}%
%    }%
%    \vskip0pt plus 1filll\relax
%    \mbox{\zsavepos{restofframe-\arabic{restofframe}-end}}%
%    \par
%}


\usepackage{tikz}
\usetikzlibrary{arrows}

%\usepackage[usenames,dvipsnames]{xcolor}
\usepackage{tkz-berge}
\usetikzlibrary{fit,shapes}

\usepackage{calc}
%%
%% The tikz package is used for doing the actual drawing.
%\usepackage{tikz}
%%
%% In order to be able to put arrowheads in the middle of directed edges, we need an extra library.
\usetikzlibrary{decorations.markings}
%%
%% The next line says how the "vertex" style of nodes should look: drawn as small circles.
\tikzstyle{vertex}=[circle, draw, inner sep=0pt, minimum size=6pt]
%%
%% Next, we make a \vertex command as a shorthand in place of \node[vertex} to get that style.
\newcommand{\vertex}{\node[vertex]}
%%
%% Finally, we declare a "counter", which is what LaTeX calls an integer variable, for use in
%% the calculations of angles for evenly spacing vertices in circular arrangements.
\newcounter{Angle}

\newtheoremstyle{example}
{\topsep} % space above
{\topsep} % space below
{} % body font
{} % indent
{\bf} % head font
{:} % punctuation between head and body
{0.5em} % space after head
{} % manually specify head
%{\thmname{#1}\thmnumber{ #2}\thmnote{:#3}} % manually specify head

\theoremstyle{example}
\newtheorem{ex}{Example}[section]

\newtheoremstyle{definition}
{\topsep} % space above
{\topsep} % space below
{} % body font
{} % indent
{\sc} % head font
{:} % punctuation between head and body
{0.5em} % space after head
{} % manually specify head
%{\thmname{#1}\thmnumber{ #2}\thmnote{:#3}} % manually specify head

\theoremstyle{definition}
\newtheorem{defn}{Definition}[section]

\theoremstyle{rem}
\newtheorem{rem}{Remark}[section]

\newtheoremstyle{theorem}
{\topsep} % space above
{\topsep} % space below
{} % body font
{} % indent
{\sc} % head font
{:} % punctuation between head and body
{0.5em} % space after head
{} % manually specify head
%{\thmname{#1}\thmnumber{ #2}\thmnote{:#3}} % manually specify head

\theoremstyle{theorm}
\newtheorem{thm}{Theorem}[section]



%%%to add in new counter for slides in beamer

%\setbeamertemplate{footline}{
%  \leavevmode%
%  \hbox{%
%  \begin{beamercolorbox}[wd=.333333\paperwidth,ht=2.25ex,dp=1ex,center]{author in head/foot}%
%    \usebeamerfont{author in head/foot}\insertshortauthor~~(\insertshortinstitute)
%  \end{beamercolorbox}%
%  \begin{beamercolorbox}[wd=.333333\paperwidth,ht=2.25ex,dp=1ex,center]{title in head/foot}%
%    \usebeamerfont{title in head/foot}\insertshorttitle
%  \end{beamercolorbox}%
%  \begin{beamercolorbox}[wd=.333333\paperwidth,ht=2.25ex,dp=1ex,right]{date in head/foot}%
%    \usebeamerfont{date in head/foot}\insertshortdate{}\hspace*{2em}
%    \insertframenumber{} \hspace*{2ex} % hier hat's sich ge�ndert
%  \end{beamercolorbox}}%
%  \vskip0pt%
%}



%%%%%

\newcommand*\oldmacro{}
\let\oldmacro\insertshortauthor
\renewcommand*\insertshortauthor{
  \leftskip=.3cm
\insertframenumber\,/\,\inserttotalframenumber\hfill\oldmacro}




%\excludecomment{notbeamer}
%\includecomment{beamer}



\title{Teaching Bayes: The Essential Parts}
\author{Rebecca C. Steorts \\ ISBA World Meeting 2016}
\date{Lecture 2: Intro to Gibbs Sampling}

\begin{document}

\maketitle



\frame{
\frametitle{Intro to Markov chain Monte Carlo (MCMC)}

Goal: sample from $f(x)$, or approximate $E_f[h(X)].$
\vskip 1em

Function $f(x)$ is very complicated and hard to sample from.

\begin{figure}[htbp]
\begin{center}
    \includegraphics[width=0.3\textwidth]{../figures/madel}
%\caption{default}
\label{default}
\end{center}
\end{figure}

}

\frame{
\vskip 1em

How to deal with this?
\begin{enumerate}
\item What's a simple way?
\item What are two other ways? 
\item What happens in high dimensions?
\end{enumerate}


}

%\frame{
%\begin{itemize}
%\item In low dimensions, IS and RS work pretty well. 
%\item But in high dimensions, a proposal $g(x)$ that worked in 2-D, often doesn't mean that it 
%will work in any dimension. 
%\item Hard to capture high dimensional spaces! 
%\item We turn to MCMC.
%\end{itemize}
%
%
%}


\frame{
\frametitle{High dimensional spaces}
\begin{itemize}
\item In low dimensions, importance and rejection sampling work pretty well. 
\item But in high dimensions, a proposal $g(x)$ that worked in 2-D, often doesn't mean that it 
will work in any dimension. 
\item Why? It's hard to capture high dimensional spaces! 
\end{itemize}

\begin{figure}
  \begin{center}
    \includegraphics[width=0.3\textwidth]{../figures/eye}
    \caption{A high dimensional space (many images). }
    \end{center}
\end{figure}

We turn to Markov chain Monte Carlo (MCMC).

}



\frame{
\frametitle{Intution}

Imagine that we have a complicated function $f$ below and it's high probability regions are represented in green.

\begin{figure}
  \begin{center}
    \includegraphics[width=0.7\textwidth]{../figures/markovChain}
    \caption{Example of a Markov chain}
    \end{center}
\end{figure}

}

\frame{
\frametitle{Intution}



\begin{figure}
  \begin{center}
    \includegraphics[width=0.7\textwidth]{../figures/markovChainPoint}
    \caption{Example of a Markov chain and red starting point}
    \end{center}
\end{figure}

}

\frame{
\frametitle{Intution}

\begin{figure}
  \begin{center}
    \includegraphics[width=0.7\textwidth]{../figures/markovChainMove}
    \caption{Example of a Markov chain and moving from the starting point to a high probability region.}
    \end{center}
\end{figure}

}



\frame{
\frametitle{What is Markov Chain Monte Carlo}

\begin{itemize}
\item Markov Chain -- where we go next only depends on our last state (the Markov property).
\item Monte Carlo -- just simulating data. 
\end{itemize}



}



\frame{
\frametitle{Why MCMC?}

\begin{itemize}
\item[(a)] the region of high probability tends to be ``connected''
\item That is, we can get from one point to another without going through a low-probability region, and
\item[(b)] we tend to be interested in the expectations of functions that are relatively smooth and have lots of ``symmetries''
\item  That is, one only needs to evaluate them at a small number of representative points in order to get the general picture.
\end{itemize}

}


\frame{
\frametitle{Advantages/Disadvantages of MCMC:}
Advantages:
\begin{itemize}
\item applicable even when we can't directly draw samples
\item works for complicated distributions in high-dimensional spaces, even when we don't know where the regions of high probability are
\item relatively easy to implement
\item fairly reliable
\end{itemize}

{Disadvantages:}
\begin{itemize}
\item slower than simple Monte Carlo or importance sampling (i.e., requires more samples for the same level of accuracy)
\item can be very difficult to assess accuracy and evaluate convergence, even empirically
% \item theoretical guarantees on convergence rates are difficult to obtain
\end{itemize}

}



\frame{
\frametitle{Two-stage Gibbs sampler}
\begin{itemize}
\item Suppose $p(x,y)$ is a p.d.f.\ or p.m.f.\ that is difficult to sample from directly.  
\item Suppose, though, that we \textit{can} easily sample from the conditional distributions $p(x|y)$ and $p(y|x)$. 
\item  The Gibbs sampler proceeds as follows: 
\begin{enumerate}
\item set $x$ and $y$ to some initial starting values
\item then sample $x|y$, then sample $y|x$,\\ then $x|y$, and so on.
\end{enumerate}
\end{itemize}

}


\frame{
\frametitle{Two-stage Gibbs sampler}
\begin{enumerate}
\item[0.] Set \textcolor{blue}{$(x_0,y_0)$} to some starting value.
\item[1.] Sample $x_1\sim p(x|y_0)$, that is, from the conditional distribution $X\mid Y=y_0$. \\
\textcolor{blue}{Current state: $(x_1, y_0)$}\\
          Sample $y_1\sim p(y|x_1)$, that is, from the conditional distribution $Y\mid X=x_1$.\\
    \textcolor{blue}{      Current state: $(x_1, y_1)$}\\
\item[2.] Sample $x_2\sim p(x|y_1)$, that is, from the conditional distribution $X\mid Y=y_1$. \\
    \textcolor{blue}{      Current state: $(x_2, y_1)$}\\
          Sample $y_2\sim p(y|x_2)$, that is, from the conditional distribution $Y\mid X=x_2$. \\
            \textcolor{blue}{      Current state: $(x_2, y_2)$}\\
        $\vdots$
\end{enumerate}
Repeat iterations 1 and 2, M times. 

This procedure defines a sequence of pairs of random variables
$$ (X_0,Y_0), (X_1,Y_1), (X_2,Y_2), (X_3,Y_3), \ldots$$

}

\frame{
\frametitle{Markov chain and dependence}


$$ (X_0,Y_0), (X_1,Y_1), (X_2,Y_2), (X_3,Y_3), \ldots$$ satisfies the property of being a Markov chain. 

\vskip 1em

The conditional distribution of $(X_i,Y_i)$ given all of the previous pairs depends only on $(X_{i-1},Y_{i-1})$

\vskip 1em

$(X_0,Y_0), (X_1,Y_1), (X_2,Y_2), (X_3,Y_3), \ldots$ are not iid samples (Think about why). 

}

\frame{
\frametitle{Ideal Properties of MCMC}

\begin{itemize}
\item $(x_0,y_0)$ chosen to be in a region of high probability under $p(x,y)$, but often this
is not so easy. 
%\item Instead, we run the chain for a long time. 
\item We run the chain for M iterations and discard the first $B$ samples $(X_1,Y_1),\ldots,(X_B,Y_B)$. This is called \emph{burn-in}.
% future: We will discuss how to choose $B$ in ???
%\item When using a burn-in period, the choice of starting point it is not particularly
%important---a poor choice will simply require a longer burn-in period.
\item Typically: if you run the chain long enough, the choice of $B$ doesn't matter. 

\item Roughly speaking, the performance of an MCMC algorithm---that is, how quickly the sample averages $\frac{1}{N}\sum_{i = 1}^N h(X_i,Y_i)$
converge---is referred to as the \emph{mixing rate}. 
\item An algorithm with good performance is said to ``have good mixing'', or ``mix well''.

\end{itemize}


}

\frame{
\frametitle{Toy Example}

Suppose we want to sample from the bivariate distribution:
$$p(x,y) \propto e^{-x y}\I(x,y\in (0, c))$$
where $c>0$, and $(0,c)$ denotes the (open) interval between $0$ and $c$. 
(This example is due to Casella \& George, 1992.)
}

\frame{
\frametitle{Toy Example}
\begin{itemize}
\item The Gibbs sampling approach is to alternately sample from $p(x|y)$ and $p(y|x)$.  
\item Note $p(x,y)$ is
symmetric with respect to $x$ and $y$.
\item Hence, only need to derive one of these and then we can get the other one by just swapping $x$ and $y$.
\item Let's look at $p(x|y).$
\end{itemize}
}

\frame{
\frametitle{Toy Example}
$$p(x,y) \propto e^{-x y}\I(x,y\in (0, c))$$

$$p(x|y) \underset{x}{\propto} p(x,y) \underset{x}{\propto} e^{-x y}\I(0<x<c)\underset{x}{\propto} \Exp(x|y)\I(x<c).\footnote{Under $\propto$, we write the random variable ($x$) for clarity.}
$$\begin{itemize}
\item $p(x|y)$ is a \emph{truncated} version of the $\Exp(y)$ distribution
\item It is the same as taking $X\sim\Exp(y)$ and
 conditioning on it being less than $c$, i.e., $X\mid X<c$.
\item Let's refer to this as the $\TExp(y,(0,c))$ distribution.
\end{itemize}
}

\frame{
\frametitle{Toy Example}
An easy way to generate a sample from $Z\sim\TExp(\theta,(0,c))$, is:
\begin{enumerate}
    \item Sample $U\sim \Uniform(0,F(c|\theta))$ where $$F(x|\theta) = 1-e^{-\theta x}$$ is the $\Exp(\theta)$ c.d.f.
    \item Set $Z = F^{-1}(U|\theta)$ where $$F^{-1}(u|\theta) = -(1/\theta)\log(1 - u)$$ is the inverse c.d.f.\ for $u\in(0,1)$.
\end{enumerate}
%To verify the last step: apply the rejection principle (along with the inverse cdf technique). 
%Do this on your own.
Verify the last step on your own.

}


\frame{

Let's apply Gibbs sampling, denoting $S=(0,c)$.
\begin{enumerate}
    \item[0.] Initialize $x_0,y_0\in S$.
    \item[1.] Sample $x_1\sim \TExp(y_0,S)$, then sample $y_1\sim \TExp(x_1,S).$
    \item[2.] Sample $x_2\sim \TExp(y_1,S)$, then sample $y_2\sim \TExp(x_2,S).$\\
        $\vdots$
    \item[$N$.] Sample $x_N\sim \TExp(y_{N-1},S)$, sample $y_N\sim \TExp(x_N,S).$
\end{enumerate}
Figure \ref{figure:toy} demonstrates the algorithm, with $c = 2$ and initial point $(x_0, y_0) = (1, 1)$.

}

\frame{

\begin{figure}
  \begin{center}
    \includegraphics[width=0.48\textwidth]{examples/toy-numbered.png}
    \includegraphics[width=0.48\textwidth]{examples/toy-scatter.png}
  \end{center}
  \caption{(Left) Schematic representation of the first 5 Gibbs sampling iterations/sweeps/scans. (Right) Scatterplot of samples from $10^4$ Gibbs sampling iterations.}
  \label{figure:toy}
\end{figure}



}

%\frame{
%\frametitle{Example: Normal with semi-conjugate prior}
%
%Consider
%$X_1,\ldots,X_n|\mu,\lambda\,\iid \N(\mu,\lambda^{-1})$.
%Then independently consider
%\begin{align*}
%& \bm\mu \sim \N(\mu_0,\lambda_0^{-1})\\
%& \bm\lambda \sim \Ga(a,b)
%\end{align*}
%
%This is called a semi-conjugate situation, in the sense that the prior on $\mu$ is conjugate for each fixed value of $\lambda$, and the prior on $\lambda$ is conjugate for each fixed value of $\mu$.
%
%\vskip 1em
%
%For ease of notation, denote the observed data points by $x_{1:n}.$
%
%}
%
%\frame{
%
%
%We know that for the Normal--Normal model, we know that for any fixed value of $\lambda$,
%$$\bm\mu|\lambda,x_{1:n}\, \sim \,\N(M_\lambda,L_\lambda^{-1})$$
% where $$L_\lambda =\lambda_0+ n\lambda \;\;  \text{and} \;\;
% M_\lambda =\frac{\lambda_0\mu_0+\lambda\sum_{i = 1}^n x_i}{\lambda_0+ n\lambda}.$$
%\pause
%For any fixed value of $\mu$, it is straightforward to derive\footnote{do this on your own} that
%\begin{align}\label{equation:lambda-semi-conjugate}
%\bm\lambda|\mu, x_{1:n}\,\sim\,\Ga(A_\mu, B_\mu)
%\end{align}
%where $A_\mu = a + n/2$ and
%$$ B_\mu = b +\tfrac{1}{2}\textstyle\sum (x_i -\mu)^2 = n\hat\sigma^2 + n (\bar x-\mu)^2$$
%where $\hat\sigma^2 = \frac{1}{n}\sum (x_i -\bar x)^2$. 
%
%
%
%
%}
%
%\frame{
%
%To implement Gibbs sampling in this example, each iteration consists of sampling:
%\begin{align*}
%    & \bm\mu|\lambda,x_{1:n}\, \sim \,\N(M_\lambda,L_\lambda^{-1})\\
%    & \bm\lambda|\mu, x_{1:n}\,\sim\,\Ga(A_\mu, B_\mu).
%\end{align*}
%
%
%}

\frame{
\frametitle{Pareto example}

Distributions of sizes and frequencies often tend to follow a ``power law'' distribution. 
\begin{itemize}
    \item wealth of individuals
    \item size of oil reserves
    \item size of cities
    \item word frequency
    \item returns on stocks
\end{itemize}
}
\frame{
\frametitle{Power Law Distribution}
The Pareto distribution with shape $\alpha>0$ and scale $c >0$ has p.d.f.\
$$ \Pareto(x|\alpha,c) = \frac{\alpha c^\alpha}{x^{\alpha+1}}\I(x>c)\propto \frac{1}{x^{\alpha+1}}\I(x>c).$$
This is referred to as a power law distribution, because the p.d.f.\ is proportional to $x$ raised to a power.
Notice that $c$ is a lower bound on the observed values.
In this example, we'll see how Gibbs sampling can be used to perform inference for $\alpha$ and $c$. 


}

\frame{

\begin{table}
\small
\centering
\begin{tabular}{clr}
Rank & City & Population \\
\hline
1  &  Charlotte   &  731424 \\
2  &  Raleigh   &  403892 \\
3  &  Greensboro   &  269666 \\
4  &  Durham   &  228330 \\
5  &  Winston-Salem   &  229618 \\
6  &  Fayetteville   &  200564 \\
7  &  Cary  &  135234 \\
8  &  Wilmington   &  106476 \\
9  &  High Point  &  104371 \\
10  &  Greenville   &  84554 \\
11  &  Asheville   &  85712 \\
12  &  Concord   &  79066 \\
\vdots  &  \vdots   &  \vdots \\
44  &  Havelock  &  20735 \\
45  &  Carrboro  &  19582 \\
46  &  Shelby   &  20323 \\
47  &  Clemmons  &  18627 \\
48  &  Lexington   &  18931 \\
49  &  Elizabeth City   &  18683 \\
50  &  Boone   &  17122 \\
\hline
\end{tabular}
\vspace{1em}
\caption{Populations of the 50 largest cities in the state of North Carolina, USA.}
\label{table:cities}
\end{table}



}

\frame{
\frametitle{Parameter Interpretations}

\begin{itemize}
\item $\alpha$ tells us the scaling relationship between the size of cities and their probability of occurring. 
\begin{itemize}
\item Let $\alpha = 1$. 
\item 
Density looks like $1/x^{\alpha +1} = 1/x^2$.
\item Cities with 10,000--20,000 inhabitants occur roughly $10^{\alpha+1} = 100$
    times as frequently as cities with 100,000--110,000 inhabitants.
%    (or $10^{\alpha +1}/10 = 10$ times as frequently as cities with
%    100,000--200,000 inhabitants).
\end{itemize}
\item $c$ represents the cutoff point---any cities smaller than this were not included in the dataset.
\end{itemize}
}


\frame{
To keep things as simple as possible, let's use an (improper) default prior:
$$p(\alpha,c) \propto \I(\alpha,c>0).$$

\vskip 1em
Recall from Module 4:
\begin{itemize}
\item An \emph{improper/default prior} is a nonnegative function of the parameters which integrates to infinity.
\item  Often (but not always!)\ the resulting ``posterior'' will be proper.
\item It is important that the ``posterior'' be proper, since otherwise the whole Bayesian framework breaks down.
\end{itemize}



}

\frame{
Recall
\textcolor{blue}{
\begin{align}
p(x|\alpha,c) &= \frac{\alpha c^\alpha}{x^{\alpha+1}}\I(x>c)\\
&\I(\alpha,c>0)
\end{align}
}
Let's derive the posterior:
\begin{align}
    p(\alpha,c|x_{1:n}) & \overset{\text{def}}{\underset{\alpha,c}{\propto}} p(x_{1:n}|\alpha,c)p(\alpha,c)\notag\\ 
    &\underset{\alpha,c}{\propto}\I(\alpha,c>0)\prod_{i=1}^n \frac{\alpha c^\alpha}{x_i^{\alpha+1}}\I(x_i>c) \notag\\
    & = \frac{\alpha^n c^{n\alpha}}{(\prod x_i)^{\alpha+1}} \I(c<x_*)\I(\alpha,c>0)\label{equation:Pareto-posterior}
                        % & = \alpha^n \Big(\prod_{i=1}^n c/x_i\Big)^\alpha \I(c<x_*)
\end{align}
where $x_* = \min\{x_1,\ldots,x_n\}$. 
\vskip 1em
As a joint distribution on $(\alpha,c)$, 
\begin{itemize}
\item this does not seem to have a recognizable form, 
\item and it is not clear how we might sample from it directly.
\end{itemize}
}

\frame{
Let's try Gibbs sampling! 
\vskip 1emTo use Gibbs, we need to be able to sample $\alpha|c,x_{1:n}$ and $c|\alpha,x_{1:n}$. 
\vskip 1em
By Equation \ref{equation:Pareto-posterior}, we find that
\begin{align*}
p(\alpha|c,x_{1:n})&\underset{\alpha}{\propto}p(\alpha,c|x_{1:n})
\underset{\alpha}{\propto} \frac{\alpha^n c^{n\alpha}}{(\prod x_i)^\alpha}\I(\alpha>0) \\
&= \alpha^n\exp\big(-\alpha(\textstyle\sum\log x_i - n\log c)\big)\I(\alpha>0) \\
&\underset{\alpha}{\propto} \Ga\big(\alpha\,\big\vert\, n+1,\,\textstyle\sum\log x_i - n\log c\big),
\end{align*}
and
\begin{align*}
p(c|\alpha, x_{1:n})\underset{c}{\propto}p(\alpha,c|x_{1:n})
\underset{c}{\propto} c^{n\alpha}\I(0<c<x_*),
\end{align*}
which we will define to be Mono$(\alpha, x_*)$


}

\frame{
\frametitle{Defining the Mono distribution}
For $a>0$ and $b>0$, define the distribution $\Mono(a,b)$ (for monomial) with p.d.f.
$$ \Mono(x|a,b)\propto x^{a-1}\I(0<x<b). $$
Since $\int_0^b x^{a -1}d x = b^a/a$, we have
$$ \Mono(x|a,b) =\frac{a}{b^a}x^{a-1}\I(0<x<b), $$
and for $0<x<b$, the c.d.f.\ is
$$ F(x|a,b) =\int_0^x \Mono(y|a,b)d y = \frac{a}{b^a}\frac{x^a}{a} = \frac{x^a}{b^a}. $$
}

\frame{
To use the inverse c.d.f.\ technique, we solve for the inverse of $F$ on $0<x<b$:
Let $u = \frac{x^a}{b^a}$ and solve for $x.$
\begin{align}
u &= \frac{x^a}{b^a} \\
b^a u &= x^a \\ 
b u^{1/a} &= x 
\end{align}
Can sample from $\Mono(a,b)$ by drawing $U\sim \Uniform(0,1)$ and setting $X=b U^{1/a}$.\footnote{ It turns out that this is an inverse of the Pareto distribution, in the sense that if $X\sim\Pareto(\alpha,c)$ then $1/X\sim\Mono(\alpha,1/c)$. }



}

\frame{
So, in order to use the Gibbs sampling algorithm to sample from the posterior $p(\alpha,c|x_{1:n})$, we initialize $\alpha$ and $c$, and then alternately update them by sampling:
\begin{align*}
\alpha|c,x_{1:n} \,&\sim\, \Ga\big(n+1,\,\textstyle\sum\log x_i - n\log c\big) \\
c|\alpha,x_{1:n}\,&\sim\, \Mono(n\alpha+1,\,x_*).
\end{align*}
}

\frame{
\frametitle{Ways of visualizing results}
 \textbf{Traceplots}. A traceplot simply shows the sequence of samples, for instance $\alpha_1,\ldots,\alpha_N$, or $c_1,\ldots, c_N$. Traceplots are a simple but very useful way to visualize how the sampler is behaving. 
 }

\frame{
\begin{figure}[htbp]
\begin{center}
 \includegraphics[width=0.65\textwidth]{examples/Pareto-a_trace.png}
\caption{Traceplot of $\alpha$}
\label{default}
\end{center}
\end{figure}


\begin{figure}[htbp]
\begin{center}
 \includegraphics[width=0.65\textwidth]{examples/Pareto-c_trace.png}
\caption{Traceplot of c.}
\label{default}
\end{center}
\end{figure}

}

%\frame{
%
% \textbf{Scatterplot}. 
% \vskip 1em
% Scatterplot visualizes the estimated posterior distribution $p(\alpha, c|x_{1:n})$.
%  \vskip 1em
%  How can we also visualize the estimated posterior? plot(density())
%   \vskip 1em
% Boone is the smallest city, with a population of 17,122. 
%  \vskip 1em
% Posterior on $c$ is quite concentrated just under 17,122, which makes sense since $c$ represents the cutoff point in the sampling process.
%
%
%}

%\frame{
%\begin{figure}[htbp]
%\begin{center}
%\includegraphics[width=0.8\textwidth]{examples/Pareto-scatterplot.png}
%\caption{Scatterplot of samples}
%\label{default}
%\end{center}
%\end{figure}
%}



\frame{
 \textbf{Estimated density}. We are primarily interested in the posterior on $\alpha$, since it tells us the scaling relationship between the size of cities and their probability of occurring. 
 
   \vskip 1em
 By making a histogram of the samples $\alpha_1,\ldots,\alpha_N$, we can estimate the posterior density $p(\alpha|x_{1:n})$. 
 
    \vskip 1em
 The two vertical lines indicate the lower $\ell$ and upper $u$ boundaries of an (approximate) 90\% credible interval $[\ell,u]$---that is, an interval that contains 90\% of the posterior probability:
$$\Pr\big(\bm\alpha\in [\ell, u] \big\vert x_{1:n}\big) = 0.9. $$
%The interval shown here is approximate since it's based on the samples. 
%This can be computed from the samples by sorting them $\alpha_{(1)}\leq \cdots\leq\alpha_{(N)}$ and setting
%$$\ell = \alpha_{(\lfloor 0.05 N\rfloor)}  \qquad u = \alpha_{(\lceil 0.95 N\rceil)} $$
%where $\lfloor x\rfloor$ and $\lceil x\rceil$ are the floor and ceiling functions, respectively.





}


\frame{
\begin{figure}[htbp]
\begin{center}
\includegraphics[width=0.9\textwidth]{examples/Pareto-a_density.png}\caption{Estimated density of $\alpha|x_{1:n}$ with $\approx$ 90 percent credible intervals.}
\label{default}
\end{center}
\end{figure}

}

\frame{
 \textbf{Running averages}. Panel (d) shows the running average $\frac{1}{k}\sum_{i = 1}^k\alpha_i$ for $k = 1,\ldots,N$.
    \vskip 1em
  In addition to traceplots, running averages such as this are a useful heuristic for visually assessing the convergence of the Markov chain. 
     \vskip 1em
  The running average shown in this example still seems to be meandering about a bit, suggesting that the sampler needs to be run longer (but this would depend on the level of accuracy desired).


}

\frame{
\begin{figure}[htbp]
\begin{center}
 \includegraphics[width=0.8\textwidth]{examples/Pareto-a_means.png}
 \caption{Running average plot}
 \label{default}
\end{center}
\end{figure}

}

\frame{
\frametitle{Survival function}
A survival function is defined to be $$S(x) = \Pr(X>x) = 1-\Pr(X\leq x).$$
\vskip 1em


 Power law distributions are often displayed by plotting their survival function $S(x),$ on a log-log plot. 
 \vskip 1em
 Why?
  $S(x) = (c/x)^\alpha$ for the $\Pareto(\alpha,c)$ distribution and on a log-log plot this appears as a line with slope $-\alpha$.
 \vskip 1em
  The posterior survival function (or more precisely, the posterior predictive survival function), is $S(x|x_{1:n}) = \Pr(X_{n+1}>x\mid x_{1:n})$. 
  }
  
  \frame{
  Figure \ref{figure:Pareto}(e) shows an empirical estimate of the survival function (based on the empirical c.d.f., $\hat F(x) = \frac{1}{n}\sum_{i = 1}^n \I(x\geq x_i)$) along with the posterior survival function, approximated by
\begin{align}
S(x|x_{1:n}) &= \Pr(X_{n+1}>x\mid x_{1:n}) \\
&= \int \Pr(X_{n+1}>x\mid \alpha,c) p(\alpha,c|x_{1:n})d\alpha d c\\
&\approx\frac{1}{N}\sum_{i = 1}^N \Pr(X_{n+1}>x\mid \alpha_i,c_i)
=\frac{1}{N}\sum_{i = 1}^N (c_i/x)^{\alpha_i}.
\end{align}
This is computed for each $x$ in a grid of values.
\vskip 1em
[Think about why each line is true on your own].
}




\frame{
\begin{figure}[htbp]
\begin{center}
 \includegraphics[width=0.8\textwidth]{examples/Pareto-survival-function.png}
 \caption{Empirical vs posterior survival function}
 \label{figure:Pareto}
\end{center}
\end{figure}


}




\frame{
\frametitle{Questions you should be able to answer!}
\begin{itemize}
\item When should we use MCMC in a Bayesian setting? 
\item When would we use an MCMC over Importance sampling
and Rejection sampling? 
\item What is a Gibbs sampler?  
\item What are simple diagnostics of MCMC?
\item Are we guaranteed convergence of the Markov chain emprically?
\item What do are diagnostics really tell us? 
\end{itemize}



}



\end{document}